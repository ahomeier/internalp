\section{Introduction}
This study examines the contribution of transient sources to the astrophysical
neutrino flux measured by IceCube (HESE flux). The special focus will be
placed upon Gamma Ray Bursts (GRBs). 

Contrary to most studies which use external GRB triggers from satellites like
Swift and Fermi, this result is based on the opposite idea to use neutrinos in 
order to trigger follow-up observations with Swift and optical telescopes. The
disadvantage of this approach is the additional background while the advantage
is the $2\pi$ field of view (FoV) (northern sky only) and the independence of 
any
gamma-ray sensitivity of the satellites.
Instead of using real measured GRBs, a Toy Monte Carlo was written to simulate
the neutrino flux from a GRB population up to a redshift of eight. The GRB
population and luminosity function is mainly based on a model by Wanderman and
Piran (section \ref{sec:GRB_models}). Other models are included to study the
model dependency.

The neutrino signal will be studied on the level of the Optical- and X-Ray
Follow Up - O(X)FU - and compared to the measured data. The OFU is described
in more detail in section \ref{sec:OFU}. 

The GRB population model, simulated GRB neutrinos and experimental results will
be combined within the analysis frame work of the GRB population Toy Monte
Carlo. It is explained in more detail in section \ref{sec:GRBTMC} while the main
results are summarized and discussed in section \ref{sec:Results}.