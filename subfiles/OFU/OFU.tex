\section{Optical- and X-Ray Follow Up}
\label{sec:OFU}
The analysis is based on data from the X-Ray Follow Up including the seasons 
IC86-1 to IC86-3. It is a system that analyzes data at the South Pole, 
searching for multiple neutrinos within 100 seconds from one direction (and a 
possible source 
location) to trigger Follow-Up observations with various telescopes including 
the XRT on board the Swift satellite. The latency of such triggers is about 
three minutes on average. The X-ray Follow-Up (XFU) is a subsystem of the 
Optical Follow-Up (OFU).
\subsection{Alert system}

Triggered events are reconstructed at Pole and evaluated at various levels. 
They are first roughly classified as possible muons by the Muon Filter 
($\sim$40 Hz). Stricter cuts are applied on the OnlineL2Filter, to filter 
out badly reconstructed events achieving a rate of about 5 Hz. The lower rate 
allows for further more detailed reconstructions which are then used for the 
OFU filter. 

The OFU filter is applied to select well reconstructed muon neutrino events 
from the northern sky. Depending on the season a purity of up to 90\% was 
achieved. The filter was improved season to season though the OFU seasons do 
not match the IceCube seasons. In the first year with the full detector 
configuration (IC86-1), the OFU system was shut down for several month as the 
OFU filter didn't work as expected. The debugging took a long time due to 
multiple software changes that were introduced for IC86-1. For the third OFU 
season in this analysis a boosted decision tree (BDT) was introduced during the 
second full IceCube season (IC86-2). As no changes were introduced for IC86-3, 
the BDT season encompasses the end of IC86-2 and the full IC86-3 IceCube 
season. The first and last runs of all seasons are listed in Table 
\ref{tab:ofu_seasons}.

\begin{table}[h]
  \centering
  \begin{tabular}{l||c|c|c}
   Season & IC86-1 & IC86-2 & IC86-BDT \\
\hline
   first run & 118691  & 120156  & 121788  \\
   start date & 2011-09-16 & 2012-05-15 & 2013-02-01\\
\hline
   last run  & 120155 & 121787 & 124701 \\
   end date & 2012-05-15  & 2013-02-01 & 2014-05-06 \\
  \end{tabular}
  \caption{The table lists the different seasons that were analyzed for this 
work.}
  \label{tab:ofu_seasons}
\end{table}

The OFU cut selection for the IC86-1 and IC86-2 seasons are based on the 
following logic
\begin{equation}
\begin{align}
Zenith (MPE) \geq 90^\circ \; \; and \; \; \frac{logl (MuonLLh)}{NCh - 3.5} & 
<= 8 \; \; and \\ \text{min} (Zenith_{Split}) &\geq 69.51^\circ \; 
\; and \\
((NDir (MPE) \geq 6 \; \;
and LDir (MPE) \geq 280 )& \; \;or \; \;(MuE (MPE) \geq 1e6))
\end{align}
\end{equation}

The $Zenith_{Split}$ values refer to the 
zenith values derived from the topological and time splitter of the events. MPE 
and MuonLLh refer to the different fit types used to extract the parameters. 
The cut scheme was completely reworked  for IC86-BDT to use a BDT cut 
(reference ???). 

The single neutrino events passing these cuts are further analyzed to search 
for multiplets with multiplets consisting of at least two events arriving 
within the time frame $\Delta t$ and an angular difference $\Delta \Psi$:
\begin{equation}
 \Delta T \leq 100 \text{ s} \;\;\; and \;\;\; \Delta \Psi \leq 3.5^\circ
\end{equation}

All seasons have in common that roughly 50 - 60  doublets per 
year are expected to be generated purely from the background of atmospheric 
neutrinos while the background of higher multiplets (triplets, quadruplets) is 
negligible in the order of once every 15 to 20 years. As observation time with 
the Swift satellite is precious, it was agreed to send only the seven most 
signal like doublets per year. The evaluation is based on the OFU test 
statistic (OFUtst)
\begin{align}
    \begin{split}
    \lambda = -2 \ln \mathcal{L} &= \frac{\Delta\Psi^2}{\sigma_q^2} + 2 \ln(2
\pi \sigma_q^2) \\
                              &- 2 \ln\left( 1 -
e^{\frac{-\theta_A^2}{2\sigma_w^2}}\right) 
                              + 2 \ln\left( \frac{\Delta T}{\unit[100]{s}}
\right)
    \end{split}
  \label{eq:alert-llh}
\end{align}
in which the time between the neutrinos in the multiplet is denoted as $\Delta 
T$
and their angular separation as $\Delta\Psi$. $\sigma_q^2 = \sigma_1^2 +
\sigma_2^2$ and $\sigma_w^2 = \left(1/\sigma_1^2 + 1/\sigma_2^2\right)^{-1}$
 are the combined uncertainties of the directional
reconstruction errors $\sigma_1$ and $\sigma_2$ of the two neutrino events.
$\theta_A$ corresponds to the (circularized) angular radius of the field
of view (FoV) of the follow-up telescope (set to $0.5^\circ$ for Swift). A cut 
value $\lambda_\text{cut}$ was chosen each season to filter out the best nine 
alerts (Table \ref{tab:llh cut values}). Statistically, two more alerts will be 
lost because Swift can't observe about 80\% of the sky due to the moon and the 
sun.

\begin{table}[h]
  \centering
  \begin{tabular}{l|c|c|c}
   Season & IC86-1 & IC86-2 & IC86-BDT \\
\hline
   cut value & -8.8  & -8.8 & -9.41 \\
  \end{tabular}
  \caption{The cut values on the likelihood to filter down the number of
background doublets to less or equal than 9.}
  \label{tab:llh cut values}
\end{table}
The full derivation of the test statistic can be found in the proposal to the 
IceCube collaboration (reference ???).


\subsection{Monitoring}
The alert system described in the previous section delivers alerts only if it 
is operational. To monitor the system an additional kind of alerts were 
introduced.
Test 
alerts are multiplets formed by neutrino events with a time 
difference of less than 100 seconds and an angular difference of $3.5 < \Delta 
\Psi \leq 7.5^\circ$. The events going into the multiplet decision have been 
selected by applying softer cuts on the Online Level 2 events, i.e. they are 
more background contaminated than the OFU events. The less stringent cuts and 
the different condition for the angular difference yield the desired test alert 
rate of about one test alert every ten minutes and prevent test alerts to be 
alerts as well. A lack of test alerts indicates problems with either the 
IceCube detector itself, our alert system or the communication processes and an 
alarm is raised after a two hour signal silence.