\subsection{Shifting Event Positions to GRB position}
\label{sec:shift2source}
In later steps the directions of different neutrino events will be compared to
each other. Therefore, all events within a zenith band are shifted such
that their true
direction  will coincide with the GRB direction and the
shifted or new reconstructed direction keep the same distance
and direction to the GRB direction.
% to $\vec{g}$ as the originally reconstructed direction $\vec{r}$ had to
% $\vec{t}$.



% In later steps the directions of different neutrino events will be compared to
% each other. Therefore, all events within a zenith band need to be shifted such
% that their true
% direction $\vec{t}$ will coincide with the GRB direction $\vec{g}$ and the
% shifted or new reconstructed direction $\vec{n}$ should have the same distance
% and direction
% to $\vec{g}$ as the originally reconstructed direction $\vec{r}$ had to
% $\vec{t}$.
% 
% The following calculations will be made in cartesian coordinates by
% transforming the zenith and azimuth angle
% \begin{equation}
%  \begin{align}
%   x &= \text{sin}\theta \cdot \text{cos} \phi \\
%   y &= \text{sin}\theta \cdot \text{sin} \phi \\
%   z &= \text{cos}\theta\\
%  \end{align}
% \end{equation}
% with $\phi \in [0, 2 \pi)$, $\theta \in [0, \pi]$
% The angular difference between $\vec{t}$ and $\vec{r}$ is
% \begin{equation}
%  \text{cos} \alpha = \frac{\vec{r} \cdot \vec{t}}{|\vec{r}| |\vec{t}|}
% \end{equation}
% and the direction of $\vec{r}$ relative to $\vec{t}$ is
% \begin{equation}
%  \vec{e} = \frac{\vec{r} - \vec{t}}{|\vec{r} - \vec{t}|}
% \end{equation}
% Therefore, the following conditions need to be met
% \begin{align}
%  \text{cos} \alpha &= \frac{\vec{r} \cdot \vec{t}}{|\vec{r}| |\vec{t}|} =
% \frac{\vec{n} \cdot \vec{g}}{|\vec{n}| |\vec{g}|} \\
% \vec{n} & = \vec{g} + c \vec{e}
% \end{align}
% leaving the factor c to be the only unkown to determine $\vec{n}$. There should
% be two solutions, one in the positive and one in the negative direction
% yealding two possible vectors that fullfill the same angular distance to the GRB
% direction $\vec{g}$. As $\vec{e}$ points into the intended direction, $c$ will
% be always chosen as positive.
% 
% Combining the conditions, one can derive the factor $c$
% \begin{equation}
% \label{eq:ang_dist_shifting}
%  \begin{align}
%  \text{cos} \alpha &= \frac{\vec{n} \cdot \vec{g}}{|\vec{n}| |\vec{g}|} \\
% &= \frac{\left(\vec{g} + c \vec{e} \right) \cdot \vec{g}}{|\vec{g} + c
% \vec{e}| |\vec{g}|} \\
% &= \frac{\left(g_x + c \cdot e_x\right) \cdot g_x + \left(g_y + c \cdot
% e_y\right) \cdot g_y + \left(g_z + c \cdot e_z\right) \cdot
% g_z}{\sqrt{\left(g_x + c e_x \right)^2 + \left(g_y + c e_y \right)^2 +\left(g_z
% + c e_z \right)^2 } \cdot \sqrt{g_x^2 + g_y^2 +g_z^2}} \\
% &= \frac{g_x^2 + g_y^2 +g_z^2 + c \cdot \left(e_x g_x + e_y g_y + e_z g_z
% \right)}{\sqrt{g_x^2 + g_y^2 +g_z^2 + 2 c \cdot \left(e_x g_x + e_y g_y + e_z
% g_z \right) + c^2 \left( e_x^2 + e_y^2 + e_z^2 \right) } \cdot \sqrt{g_x^2 +
% g_y^2 +g_z^2}} \\
% &= \frac{\gamma + c \cdot \tau}{\sqrt{\gamma + 2 c \cdot \tau + c^2 \cdot \zeta}
% \sqrt{\gamma}}
%  \end{align}
% \end{equation}
% with the abbreviations
% \begin{equation}
%  \begin{align}
%   \gamma &= \vec{g}^2 = g_x^2 + g_y^2 +g_z^2 \\
%   \tau &= \vec{g}\cdot \vec{e}=e_x g_x + e_y g_y + e_z g_z\\
%   \zeta &=  \vec{e}^2 =e_x^2 + e_y^2 + e_z^2
%  \end{align}
% \end{equation}
% Calculating the square of equation \ref{eq:ang_dist_shifting} and rewriting it
% to fit the
% standard p,q - formulism, one obtains
% \begin{equation}
%  \begin{align}
% \left(
% \gamma + 2 c \cdot \tau + c^2 \cdot \zeta \right) \gamma \cdot \text{cos}^2
% \alpha - \left(\gamma^2 + 2 c \tau \gamma  + c^2 \tau^2 \right) &= 0\\
% \text{cos}^2\alpha \left(\gamma^2 + 2 c \cdot \tau \gamma + c^2 \cdot \zeta
% \gamma \right) - \left( \gamma^2 + 2 c \tau \gamma  + c^2 \tau^2\right) &= 0\\
% c^2 \cdot \left(\zeta \gamma \text{cos}^2\alpha - \tau^2 \right) + 2 c \cdot
% \tau \gamma \left( \text{cos}^2\alpha - 1 \right) + \gamma^2 \left(
% \text{cos}^2\alpha - 1 \right)  &= 0 \\
% c^2 + c \cdot \frac{ 2 \tau \gamma \left(\text{cos}^2\alpha - 1 \right)}{\zeta
% \gamma \text{cos}^2\alpha - \tau^2} + \frac{\gamma^2 \left(
% \text{cos}^2\alpha - 1 \right) }{\zeta
% \gamma \text{cos}^2\alpha - \tau^2} &= 0 \\
% c^2 + p \cdot c + q &= 0
% %  \end{align}
% %  \begin{align}
%  \end{align}
% \end{equation}
% Therefore, $c$ has two solutions as predicted. The positive will always be
% chosen.
% \begin{equation}
%  c = -  \frac{\tau \gamma \left(\text{cos}^2\alpha - 1 \right)}{\zeta
% \gamma \text{cos}^2\alpha - \tau^2} \pm \sqrt{\left( \frac{\tau \gamma
% \left(\text{cos}^2\alpha - 1 \right)}{\zeta
% \gamma \text{cos}^2\alpha - \tau^2} \right)^2 - \frac{\gamma^2 \left(
% \text{cos}^2\alpha - 1 \right) }{\zeta
% \gamma \text{cos}^2\alpha - \tau^2}}
% \end{equation}
% 
% The value of c leads to the calculation of the new reconstructed direction
% $\vec{n}$ which can be transformed back into spherical coordinates using
% \begin{equation}
%  \theta = \text{arccos} \left( \frac{n_z}{|\vec{n}|} \right)
% \end{equation}
% \begin{equation}
%  \begin{align}
%   \phi &= \text{arctan} \left( \frac{n_y}{n_x}\right)  \text{,  if  } n_x > 0\\
%   \phi &= sign \left(n_y \right) \cdot \frac{\pi}{2}  \text{,  if  } n_x = 0\\
%   \phi &= \text{arctan} \left( \frac{n_y}{n_x}\right) + \pi  \text{,  if  } n_x
% <0 \text{ and } n_y \geq 0\\
%   \phi &= \text{arctan} \left( \frac{n_y}{n_x}\right) - \pi  \text{,  else}\\
%  \end{align}
% \end{equation}
% 
% 
% % \begin{framed}
% % There are very few events for which this is
% % not true. The reason is unknown at this moment. 
% % \end{framed}
%  
% 
% 
%  
% \begin{figure}[htbp]
%   \centering
% \includegraphics[width=
% 1.\textwidth]{fig/shift2source_true_minus_shifted_error.pdf}
%   \caption{\label{fig:shift2source_proof}Shown is the difference between the
% true error between reconstructed and true direction and the error between the
% shifted reconstructed direction and the GRB direction. Almost all events have 
% been shifted perfectly.}
% \end{figure}
% 
% \begin{figure}[h]
% \centering
%  \captionsetup{width=.9\textwidth}
% %  \captionsetup{margin=0pt}
% \subfloat[Dependency on zenith angle.\label{fig:shift2source_zendependency}]{%
%  \includegraphics[width=0.45\textwidth]{fig/shift2source_zendependency.pdf}}
%  \subfloat[Dependency on the azimuth angle. 
% \label{fig:shift2source_azidependency}]{%
% \includegraphics[width=0.45\textwidth]{fig/shift2source_azidependency.pdf}}
% \caption{Two dimensional histograms showing the dependency of the difference 
% between shifted and true reconstruction error to the zenith (left) and azimuth 
% angle (right). No true dependency can be seen.}
% \end{figure}
% 
% Figure \ref{fig:shift2source_proof} demonstrates that, for almost all events, 
% the new directions 
% have
%  the same distance to the GRB as the originally
% reconstructed directions had compared to their true directions. There are few 
% events \textbf{(I need a percentage here!!!???)} for which the process doesn't 
% work. The reasons are unknown at this point. Figures 
% \ref{fig:shift2source_zendependency} and \ref{fig:shift2source_azidependency} 
% demonstrate that there is no directional correlation. However, as this effect 
% is only true for ???\% of the events, the effect can be neglected.
% % An example skymap for one GRB and the correspondingly shifted neutrino
% % directions is shown in Figure \ref{fig:skymap_grb}.