\subsubsection{Zenith Bands}
\label{sec:zenith_bands}
The probability to detect a neutrino is highly dependent on the zenith angle of
its origin. The number of expected neutrinos within IceCube can be
calculated for all simulated events (eq. \ref{eq:Nexp_general}). However, these
are distributed over the whole sky and might not represent a GRB from a
specific zenith direction very well. Therefore, only events from a zenith 
region around
a drawn GRB direction will be used to calculate the expected signal. The true
direction of each considered event needs to be within a range around the GRB
direction in $\text{Cos} \Theta$
\begin{equation}
\text{Cos}\left(\Theta_{\nu, \text{true}}\right) \in
\left[\text{Cos}\left(\Theta_\text{GRB}\right) - ZBW,\,
\text{Cos}\left(\Theta_\text{GRB}\right) +
ZBW\right]
\end{equation}
The zenith band width is set to $ZBW = 0.05$ and equal in cosinus of the zenith
angle to achieve similar (and enough) statistics near pole and horizon. 

%??? above horizon?

%A side effect might be that more deviation towards pole 
Consequently, the number of expected neutrino events within the detector (eq.
\ref{eq:Nexp_general}) is not calculated anymore based on the total number of
generated events $N_\text{generated}$ over the whole sky but only a fraction of
events within the zenith band. Therefore, $N_\text{generated}$ needs to be
replaced by an effective number of generated events
\begin{equation}
N_\text{generated}^\text{eff} = N_\text{generated} \cdot \frac{A_{ZB}}{4\pi}
\end{equation}
in which $A_{ZB}$ is the area of the zenith band. The number of expected
neutrinos is then 
\begin{equation}
\label{eq:Nexp_wZB}
 N_\text{exp}^\nu = \sum_i\frac{dF_P(E_i)}{dE_i} \cdot
\frac{OneWeight_i}{N_\text{generated}^\text{eff}} = 
\sum_i\frac{dF_P(E_i)}{dE_i} \cdot
\frac{OneWeight_i}{N_\text{generated} \cdot \frac{A_{ZB}}{4\pi}}
\end{equation}